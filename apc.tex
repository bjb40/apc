\documentclass[]{article}
\usepackage{lmodern}
\usepackage{amssymb,amsmath}
\usepackage{ifxetex,ifluatex}
\usepackage{fixltx2e} % provides \textsubscript
\ifnum 0\ifxetex 1\fi\ifluatex 1\fi=0 % if pdftex
  \usepackage[T1]{fontenc}
  \usepackage[utf8]{inputenc}
\else % if luatex or xelatex
  \ifxetex
    \usepackage{mathspec}
    \usepackage{xltxtra,xunicode}
  \else
    \usepackage{fontspec}
  \fi
  \defaultfontfeatures{Mapping=tex-text,Scale=MatchLowercase}
  \newcommand{\euro}{€}
\fi
% use upquote if available, for straight quotes in verbatim environments
\IfFileExists{upquote.sty}{\usepackage{upquote}}{}
% use microtype if available
\IfFileExists{microtype.sty}{%
\usepackage{microtype}
\usepackage{unicode-math}
\UseMicrotypeSet[protrusion]{basicmath} % disable protrusion for tt fonts
}{}
\ifxetex
  \usepackage[setpagesize=false, % page size defined by xetex
              unicode=false, % unicode breaks when used with xetex
              xetex]{hyperref}
\else
  \usepackage[unicode=true]{hyperref}
\fi
\usepackage[usenames,dvipsnames]{color}
\hypersetup{breaklinks=true,
            bookmarks=true,
            pdfauthor={},
            pdftitle={Approximating Age-Period-Cohort Estimates by Averaging Multiple Models with Varying Window Restrictions},
            colorlinks=true,
            citecolor=blue,
            urlcolor=blue,
            linkcolor=magenta,
            pdfborder={0 0 0}}
\urlstyle{same}  % don't use monospace font for urls
\setlength{\parindent}{0pt}
\setlength{\parskip}{6pt plus 2pt minus 1pt}
\setlength{\emergencystretch}{3em}  % prevent overfull lines
\providecommand{\tightlist}{%
  \setlength{\itemsep}{0pt}\setlength{\parskip}{0pt}}
\setcounter{secnumdepth}{0}

\title{Approximating Age-Period-Cohort Estimates by Averaging Multiple Models
with Varying Window Restrictions}
\author{true \and true}
\date{}

% Redefines (sub)paragraphs to behave more like sections
\ifx\paragraph\undefined\else
\let\oldparagraph\paragraph
\renewcommand{\paragraph}[1]{\oldparagraph{#1}\mbox{}}
\fi
\ifx\subparagraph\undefined\else
\let\oldsubparagraph\subparagraph
\renewcommand{\subparagraph}[1]{\oldsubparagraph{#1}\mbox{}}
\fi

\begin{document}
\maketitle

\section{Abstract}\label{abstract}

Key words: Computational Sociology, Methods, Life Course, Bayesian,
Culture

\section{Introduction}\label{introduction}

\section{Background}\label{background}

Both classic and modern studies have proposed different effects across
three social dimensions of time: age, period, and cohort (Ryder 1965).
These dimensions of time operate on individuals in different ways as
they flow through social space. Age effects are driven either by the
common experiences associated with the biological process of aging
(Jackson, Weale, and Weale 2003), or persistent age-structuring
institutions, like high school or retirement (Leisering 2003; Moen
2014). Period effects are responses of everyone to contemporaneous
social experiences, like recessions or wars (Lam, Fan, and Moen 2014).
Finally, cohort effects are socialization effects. They are broad-based
historical events that stick to the populations experiencing them, even
after the event has long since ended (Vaisey and Lizardo 2016). In
contemporary research, cohorts are virtually always defined in terms of
birth year.

These three dimensions of time sit at the center of some recurrent
debates. For example, cultural sociologists are divided over whether the
main driver of culture is a period process (cultural fragmentation) or a
cohort process (acquired dispositions) (Vaisey and Lizardo 2016).
Clinical reasearchers and biodemographers are looking for biological
age---indicators of biological deterioration---but argue about
confounding from cohort changes (Jackson et al. 2003). And yet others
are concerned about separating long-term and short-term impacts of
important events, like the Great Recession (Burgard and Kalousova 2015).

Unfortunately, it is not trivial to estimate the unique effects of age,
period, and cohort to find evidence that may settle these questions and
debates. The fundamental problem is one of statistical identification:
these three dimensions are linearly dependent. Two of the dimensions
define the third. If a researcher knows an individual's age, and the
year of the survey, cohort is also defined. Because these variables are
exactly colinear, they are not estimable using classic statistical
techniques. A number of solutions have been proposed to address this
conundrum, these include traditional methods like block/window
constraints (Anon. 2005), and new methods, including statistical
transformation (the ``intrinsic estimator''), and random effects models
(Yang and Land 2013).

The past several years have seen a resurgence in debates surrounding
these issues (Bell and Jones 2017; Luo 2013; Luo and Hodges 2016).
Perhaps the most common argument is that the ``constraints'',
\emph{i.e.} assumptions used to break the APC identity, produce unknown
biases into the models. In reality, this presents an extreme case of
multicollinearity, a well-known, though difficult problem in statistical
theory (Wooldridge 2009, at p.~95-98), with numerous cautions in applied
texts that separating highly colinear effects is difficult (Camm 2016,
at p.~328). From a Bayesian perspective, collinearity and
multicollinearity significantly reduce the ability of the data to supply
information to produce the estimates, potentially introducing
sensitivity to the prior (Gelman 2014, at p.~305-306), including
assumtpions of linearity. To put it another way, the critique of APC
models in general, and window constraints in particular (Luo and Hodges
2016), is that different sets of assumptions lead to different (and
inconsistent) results. But, what if Instead of focusing on finding a
\emph{best} fitting model, we use a simple nonlinear approach (block
modelling), and hundreds of assumptions? BMA allows us to do just that:
run hundreds of models with differeing assumptions, and then combine
them, weighted by their fit with the data.

\section{Bayesian Model Averaging}\label{bayesian-model-averaging}

The theoretical backdrop of BMA applies to this curcumstance quite well.
In principal, there is no one true (or best) model; instead estimates
are conditional on models from the modeling space (\(\mathscr{M}\)), and
have a posterior distribution, which is calculated as a weighted average
of all models (Raftery 1995:144--45). It has been applied in diverse
areas from weather forecasting to biology to social science (Fragoso and
Neto 2015) . BMA operates under the simple fact that any particular
estimate, including effect size and significance, have a posterior
probability distribution which is calculated as ``an average of the
posterior distributions under each of the models considered, weighted by
their posterior model probability.'' (Hoeting et al. 1999). The major
difficulties for BMA are (1) how to sample models to test, and (2) how
to calculate the posterior model probability given the data (or
\(p(M|D)\) where \(M\) is the model, and \(D\) is the data). For model
selection, we use the Markov Chain Monte Carlo Composition (MC3) method.
We use the Bayesian Information Crieterion (BIC) approximation. To
implement the MC3 method, we use the following steps:

\begin{enumerate}
\def\labelenumi{\arabic{enumi}.}
\tightlist
\item
  Define a jumping distribution \(g(.)\), so that
  \(g(M \rightarrow M')\) is non-zero for all possible window
  constraints.
\item
  Specify a starting model, \(M\), and ellicit priors for models \(M\)
  in \(\mathscr{M}\).
\item
  Given that the chain is in state \(M\), draw \(M'\), and accept it
  with probability
\end{enumerate}

\[
min  \Bigg \{ 1, \frac{p(M'|D)}{p(M|D)} \Bigg \} 
\]

otherwise, retain \(M\).

\subsection{Step One: Defining a Window Constraint
Sampler}\label{step-one-defining-a-window-constraint-sampler}

In terms of window constraints, the target model, which is inestimable,
is built by estimating each unique value of age, period, and cohort, as
a dummy variable series:

\[
E(Y) = \beta_{00} + \sum_{\lambda=2}^{\lambda} \beta_{1\lambda}A_{\lambda} +  \sum_{\rho=2}^{\rho} \beta_{2\rho}P_{\rho} +  \sum_{\kappa=2}^{\kappa} \beta_{3\kappa}C_{\kappa}
\]

where \(\beta\) is an estimated effect, \(\lambda\), \(\rho\) and
\(\kappa\) index unique values for age, period, and cohort, and \(A\),
\(P\), and \(C\) stand for matricies of dummy variable series for age,
period, and cohort. This model is unidentified, because, as with the
continuous case, the dummy variables in any two of the matricies above
fully condition the third matrix. In other words, the indicator variable
in \(C\) is a function of the indicators in \(A\) and \(P\) (in
mathematical terms,the probability that any given cohort dummy variable
is one or zero is exctly dependent on the values of A and P, so that
\(p(C_{\kappa}=0|A_{\lambda},P_{\kappa}\)) is always exaclty zero, or
exactly one, depending on the values of \(A\) and \(P\)).

We can break this dependencey, however, by transofrming \(A\), \(P\),
and \(C\)---preferably without resulting to some prespecified arbitrary
set of constraints (Gelman 2014, at p.~366). How do the window
constraints break the linear dependency? By way of example, we can
construct an age dummy variable series where \(A\) is sliced into two
groups based on some cut-point so that, for example, individuals who are
older than 30 have a dummy variable of 1 and the dummy variable for
those 30 or younger euqls 0. This would identify a binary dummy variable
with an older and a younger group. We can generalize this expression to
an arbitrary vector of cut-points, \(G\) with subscript \(\gamma\) so
that the window constraints of \(A\) in equation \_\_ are as follows:

\[
\mbox{for} \gamma < max(\gamma):
A_{\lambda}  =
\begin{cases}
  1, & \mbox{if} & a > G_{\gamma} & \mbox{and} & a \leq G_{\gamma+1} \\
  0, & \mbox{if} & a < G_{\gamma} & \mbox{or} & a > G_{\gamma+1}
\end{cases}
\]

If the vector \(G\) and \(\gamma\) have the following properties:
\(max(G) = max(a)\), \(min(G) < min(a)\), \(G_2 \geq min(a)\) and
\(\gamma>2\), then \(G\) can describe any posible sets of window
restrictions for age. The first three requirement ensures that the dummy
variable series \(A\) is fully defined across the entire range of the
continuous variable \(a\). In particular, the first restriction ensures
that the dummy variable series with the oldest ages in \(G\) contains
the maximum value of \(a\). Similarly, the next two restrictins ensure
that the smallest window constraint in \(G\) contains the minimum value
of \(a\). The final constraint requires that \(G\) have at least three
elements. Three elements in \(G\) defines a dummy variable. Using the
example above, if \(a\) has a range of 5 to 50, then the dummy variale
distinguishing older and younger respondents can be defined by
\(G \in \{4,30,50\}\).

Generalizing cross all dimensions of APC, permuting three similar
vectors (say \(G^{(d)}\)) will describe any model for any possible
window constraints detailed in equation 1. Accordingly, all permutations
of G, as defined above, constitute the model space of window constraints
(\(\mathscr{M}\)). For any given set of APC variables, \(\mathscr{M}\)
is finite, but it can become large. For example, 5 unique ages, periods,
and cohorts allow for 3,375 unique window models\footnote{A continuous
  variable of 5 integers can be sliced into continous window constraints
  in 15 ways (where \textbar{} indicates a window break for dummy
  variables):

  2 windows, 4 combinations: 1\textbar{}2345, 12\textbar{}345,
  123\textbar{}45, 1234\textbar{}5

  3 windows, 6 combinations: 1\textbar{}2\textbar{}345,
  1\textbar{}23\textbar{}45, 1\textbar{}234\textbar{}5,
  12\textbar{}3\textbar{}45, 12\textbar{}34\textbar{}5,
  123\textbar{}4\textbar{}5

  4 windows, 4 combinations: 1\textbar{}2\textbar{}3\textbar{}45,
  1\textbar{}2\textbar{}34\textbar{}5,
  1\textbar{}23\textbar{}4\textbar{}5,
  12\textbar{}3\textbar{}4\textbar{}5

  5 windows, 1 combination:
  1\textbar{}2\textbar{}3\textbar{}4\textbar{}5

  Since the hypothetical assumes 3 variables (A,P,and C) of 15
  combinations each, total combinations are \(15^3=3,375\). Similar
  calculations over an integer of 6 leads to 34 window combinations over
  each dimension for a total of 39,304 possible models.}, and 6 unique
values for each of age, period, and cohort allow for 38,304 different
models. In any model space, only 1 model is inestimable because of
perfect colinearity. This target model is (theoretially) the least
biased, although it is almost certainly the most parsimonious. The
question is how to best use information from some subset of possible
models in \(\mathscr{M}\) to estimate unbiased APC effects of the target
model. Bayesian Model Averaging (BMA) provides a straightforward way to
combine models.

There are two features of \(G\) that make the MC3 algorithm provides an
attractive way to sample models. \(G\) can be decomposed into two parts:
(1) the number of windows, and (2) the break points for each of the
windows. By disaggregating \(G\) into these two parts, we use the
Dirchelet distribution as the jumping distribution to construct matrix
\(G\) to define models.

\emph{Using the Dirichlet Distribution to Sample Window Groups (G).} As
noted above, there are two basic features of vector \(G\). First, is the
number of window breaks, or the rank of \(G\), and second is the
location of the window breaks, or the values of \(G\). We use a uniform
distribution over the range of A,P, or C to sample the number of window
breaks. This is simple, straightforward, and noninformative. We use the
Dirichelet distributon to sample the location of window breaks. The
Dirichelet is well-suited to this task, and is commonly used in
classification tasks \texttt{citation}. In additon, Taddy et al. (2015)
show that the Dirichelet is a natural prior distribution for variable
selection and value-splitting in classification and regression trees
(CART) algortithms, common for machine learning. Our approach is a
similar, as developing \(G\) is fundamentally a classificaiton task
which aggregates similar ages, periods, and cohorts.

To implement our sampling scheme, we draw two sets of auxiliary
variables for each dimension (\(d\)) of APC.

the window breaks, \(G^{(d)}\) of equation 2 are decomposed into (1) a
simplex for each dimension, with the same length of unique elements in
\(d\) (\(B^{(d)}\)), and (2) a scalar integer, \(w^d\). \(G^{(d)}\) is
simply the product of \(B\) times \(w\). We use a uniform distribution
to sample \(w\), and a Dirchelet distribution to sample from the simplex
as follows:

\[
\begin{matrix}
G^{(d)}_b = \Bigg \lfloor w^{(d)} \sum_{i=1}^{\tau_d} B^{(d)}_i \Bigg \rfloor,
      \mbox{where} G_b > G_{b-1} \mbox{and} b \subseteq \tau_d   \\
w^{(d)} \sim U(2,max(T^{(d)}))  \\
B^{(d)} \sim Dir(\alpha_{\tau_d}) \\
\end{matrix}
\]

Where \(\tau_{d}\) is the index number for the APC effects (the
\(\lambda\), \(\rho\), or \(\kappa\) of equaiton 1) \(T\) is the
continuous vector of values (the \(A\),\(P\), or \(C\)).
\texttt{add\ explanation}

\emph{Priors and Initial Model} The \(\alpha\) for the Dirichelet
distribution are for each of the vunique values in the dimension (\(d\))
for APC. The Dirchelet distribution draws a random projection on the
standard simplex, \emph{i.e.} a vector of weights \(B\) weights between
0 and 1 which sum to 1. Multiplying the vector of weights (\(B\)) by a
scalar \(w\) provides a set of numbers which sum to the scalar value
\(w^{(d)}\). We transform \(B\) and \(w\) into matrix \(G\) by selecting
the unique floor rounded values of the product of the cumulative sums of
\(B\):

Where \(G\) is a vector of the type described in equation 2, and \(d\)
indexes \(G\) and is a subset of the unique values in \(T\).

\section{A Simulation}\label{a-simulation}

\section{An Empirical Example from the
GSS}\label{an-empirical-example-from-the-gss}

\section{Discussion}\label{discussion}

\section*{Conclusion}\label{conclusion}
\addcontentsline{toc}{section}{Conclusion}

\hyperdef{}{ref-glennux5fstrategiesux5f2005}{\label{ref-glennux5fstrategiesux5f2005}}
Anon. 2005. ``Strategies for Estimating Age, Period, and Cohort
Effects.'' Pp. 12--35 in \emph{Cohort analysis}. 2455 Teller
Road,?Thousand Oaks?California?91320?United States of America? SAGE
Publications, Inc.

\hyperdef{}{ref-bellux5fhierarchicalux5f2017}{\label{ref-bellux5fhierarchicalux5f2017}}
Bell, Andrew and Kelvyn Jones. 2017. ``The Hierarchical
Age-Period-Cohort Model: Why Does It Find the Results That It Finds?''
\emph{Quality \& Quantity}.

\hyperdef{}{ref-burgardux5feffectsux5f2015}{\label{ref-burgardux5feffectsux5f2015}}
Burgard, Sarah A. and Lucie Kalousova. 2015. ``Effects of the Great
Recession: Health and Well-Being.'' \emph{Annual Review of Sociology}
41(1):181--201.

\hyperdef{}{ref-cammux5fessentialsux5f2016}{\label{ref-cammux5fessentialsux5f2016}}
Camm, Jeffrey D. 2016. \emph{Essentials of Business Analytics}. Mason,
OH: Cengage South western.

\hyperdef{}{ref-fragosoux5fbayesianux5f2015}{\label{ref-fragosoux5fbayesianux5f2015}}
Fragoso, Tiago M. and Francisco Louzada Neto. 2015. ``Bayesian Model
Averaging: A Systematic Review and Conceptual Classification.''
\emph{arXiv preprint arXiv:1509.08864}.

\hyperdef{}{ref-gelmanux5fbayesianux5f2014}{\label{ref-gelmanux5fbayesianux5f2014}}
Gelman, Andrew. 2014. \emph{Bayesian Data Analysis}. Third edition. Boca
Raton: CRC Press.

\hyperdef{}{ref-hoetingux5fbayesianux5f1999}{\label{ref-hoetingux5fbayesianux5f1999}}
Hoeting, Jennifer A., David Madigan, Adrian E. Raftery, and Chris T.
Volinsky. 1999. ``Bayesian Model Averaging: A Tutorial.''
\emph{Statistical science} 382--401.

\hyperdef{}{ref-jacksonux5fbiologicalux5f2003}{\label{ref-jacksonux5fbiologicalux5f2003}}
Jackson, Stephen HD, Martin R. Weale, and Robert A. Weale. 2003.
``Biological Age---what Is It and Can It Be Measured?'' \emph{Archives
of gerontology and geriatrics} 36(2):103--15.

\hyperdef{}{ref-lamux5fisux5f2014}{\label{ref-lamux5fisux5f2014}}
Lam, Jack, Wen Fan, and Phyllis Moen. 2014. ``Is Insecurity Worse for
Well-Being in Turbulent Times? Mental Health in Context.'' \emph{Society
and Mental Health} 4(1):55--73.

\hyperdef{}{ref-mortimerux5fgovernmentux5f2003}{\label{ref-mortimerux5fgovernmentux5f2003}}
Leisering, Lutz. 2003. ``Government and the Life Course.'' Pp. 205--25
in \emph{Handbook of the life course}, edited by Jeylan T. Mortimer and
Michael J. Shanahan. New York: Kluwer Academic/Plenum Publishers,

\hyperdef{}{ref-luoux5fassessingux5f2013}{\label{ref-luoux5fassessingux5f2013}}
Luo, Liying. 2013. ``Assessing Validity and Application Scope of the
Intrinsic Estimator Approach to the Age-Period-Cohort Problem.''
\emph{Demography} 50(6):1945--67.

\hyperdef{}{ref-luoux5fblockux5f2016}{\label{ref-luoux5fblockux5f2016}}
Luo, Liying and James S. Hodges. 2016. ``Block Constraints in
Age--Period--Cohort Models with Unequal-Width Intervals.''
\emph{Sociological Methods \& Research} 45(4):700--726.

\hyperdef{}{ref-waiteux5fconstrainedux5f2014}{\label{ref-waiteux5fconstrainedux5f2014}}
Moen, Phyllis. 2014. ``Constrained Choices: The Shifting Institutional
Contexts of Aging and the Life Course.'' Pp. 175--216 in \emph{New
directions in the sociology of aging}, edited by Linda J. Waite, Thomas
J. Plewes, and National Research Council (U.S.). Washington, D.C:
National Academies Press.

\hyperdef{}{ref-rafteryux5fbayesianux5f1995}{\label{ref-rafteryux5fbayesianux5f1995}}
Raftery, Adrian E. 1995. ``Bayesian Model Selection in Social
Research.'' \emph{Sociological Methodology} 25:111--63.

\hyperdef{}{ref-ryderux5fcohortux5f1965}{\label{ref-ryderux5fcohortux5f1965}}
Ryder, Norman B. 1965. ``The Cohort as a Concept in the Study of Social
Change.'' \emph{American Sociological Review} 30(6):843--61.

\hyperdef{}{ref-taddyux5fbayesianux5f2015}{\label{ref-taddyux5fbayesianux5f2015}}
Taddy, Matt, Chun-Sheng Chen, Jun Yu, and Mitch Wyle. 2015. ``Bayesian
and Empirical Bayesian Forests.'' \emph{arXiv preprint
arXiv:1502.02312}.

\hyperdef{}{ref-vaiseyux5fculturalux5f2016}{\label{ref-vaiseyux5fculturalux5f2016}}
Vaisey, Stephen and Omar Lizardo. 2016. ``Cultural Fragmentation or
Acquired Dispositions? A New Approach to Accounting for Patterns of
Cultural Change.'' \emph{Socius: Sociological Research for a Dynamic
World} 2:2378023116669726.

\hyperdef{}{ref-wooldridgeux5fintroductoryux5f2009}{\label{ref-wooldridgeux5fintroductoryux5f2009}}
Wooldridge, Jeffrey M. 2009. \emph{Introductory Econometrics: A Modern
Approach}. 4th ed. Mason, OH: South Western, Cengage Learning.

\hyperdef{}{ref-yangux5fage-period-cohortux5f2013}{\label{ref-yangux5fage-period-cohortux5f2013}}
Yang, Yang and Kenneth C. Land. 2013. \emph{Age-Period-Cohort Analysis:
New Models, Methods, and Empirical Applications}. Boca Raton, FL: CRC
Press.

\end{document}
